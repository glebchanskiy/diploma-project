\sectioncentered*{Введение}
\addcontentsline{toc}{section}{Введение}

В современных условиях цифровой трансформации бизнеса эффективное управление коммуникациями с пользователями становится критическим фактором успеха для организаций любого масштаба. Стремительный рост объемов пользовательских обращений, увеличение числа каналов взаимодействия и повышение ожиданий клиентов относительно скорости и качества обслуживания создают новые вызовы для традиционных систем поддержки. По данным исследований, более 70\% пользователей ожидают получить ответ на свое обращение в течение первых 30 минут, при этом около 40\% запросов являются типовыми и повторяющимися \cite{gladlyReport}.

Существующие решения в области управления обращениями зачастую характеризуются фрагментарностью, низкой адаптивностью к изменяющимся бизнес-процессам и недостаточной прозрачностью принятия решений. Большинство систем либо предлагают ограниченную автоматизацию простейших сценариев, либо работают как ``черные ящики``, не предоставляя объяснений предлагаемых решений. Это приводит к высокой нагрузке на операторов, длительному времени ожидания ответа и, как следствие, снижению удовлетворенности пользователей.

Целью данного дипломного проекта является разработка интегрированной системы управления потоками пользовательских обращений, способной  сократить время обработки типовых запросов, снизить операционные затраты и повысить качество обслуживания. В основу проектируемой системы положены принципы иерархического структурирования знаний, прозрачности и объяснимости принимаемых решений, а также возможности адаптации к различным предметным областям.

В рамках дипломного проекта решаются следующие задачи:

\begin{itemize}
\item анализ существующих подходов к автоматизации обработки пользовательских обращений и выявление их ограничений;
\item проектирование архитектуры системы, обеспечивающей возможность в автоматическом режиме отвечать на пользовательские обращений предоставляя информацию о том как было достигнуто это решение;
\item разработка интеллектуальных механизмов обработки запросов, в частности, алгоритм последовательного поиска по иерархической базе знаний;
\item реализация прототипа системы с возможностью интеграции с различными каналами коммуникации;
\item экономическое обоснование эффективности внедрения разработанного решения.
\end{itemize}

Предлагаемый подход отличается от типовых решений тем, что система не генерирует ответы на основе языковых моделей, а осуществляет навигацию по предварительно верифицированным знаниям с обязательным указанием пути, по которому был получен ответ. Это обеспечивает прозрачность процесса и позволяет пользователю понять, почему система предложила именно такое решение.

Практическая значимость работы заключается в создании универсальной платформы для автоматизации процессов обработки обращений, которая может быть адаптирована к различным предметным областям – от технической поддержки до медицинской диагностики и юридического консультирования.

Дипломный проект выполнен самостоятельно, проверен в системе "Антиплагиат". Процент оригинальности соответствует норме, установленной кафедрой ИИТ. Цитирования обозначены ссылками на публикации, указанные в "Списке использованных источников".