\sectioncentered*{Заключение}
\addcontentsline{toc}{section}{Заключение}

В ходе дипломного проекта была проведена комплексная работа по анализу, проектированию и внедрению интеллектуальной системы управления обращениями, ориентированной на автоматизацию процессов, повышение эффективности работы операторов и улучшение взаимодействия с пользователями. На этапе анализа были выявлены ключевые функциональные требования, такие как упрощение взаимодействия с пользователями, поддержка различных каналов поступления обращений, а также возможность использования интеллектуальных механизмов для улучшения качества обслуживания.

В рамках проектирования была разработана концептуальная модель системы, которая включала описание ключевых сущностей, таких как обращения, пользователи, операторы, шаблоны, роли и разрешения. Эти элементы образуют основу системы, обеспечивая её гибкость и масштабируемость. Также были разработаны диаграммы вариантов использования, которые помогли определить основные сценарии взаимодействия пользователей и операторов с системой. Важным этапом стала детальная проработка интеллектуальных механизмов, таких как прогнозирование эскалаций и автоматические рекомендации для операторов, а также использование ИИ для поиска шаблонов и предложений решений.

Кроме того, была представлена модель данных, включающая описание сущностей и их взаимосвязей, что позволило создать структуру, обеспечивающую эффективность обработки обращений и гибкость системы. Прогнозирование эскалаций и подготовка автоматических ответов на основе ИИ стали ключевыми инновационными функциями, значительно повышающими скорость и качество обработки запросов.

Результатом проделанной работы стало создание основ для дальнейшей разработки системы. Разработанные проектные материалы обеспечат возможность гибкой настройки функционала и бизнес-логики без привязки к жёсткому коду. На основе подготовленных решений можно эффективно двигаться к этапу реализации и тестирования системы, что позволит повысить удобство работы с обращениями и улучшить качество обслуживания пользователей.