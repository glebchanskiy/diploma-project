\sectioncentered*{Заключение}
\addcontentsline{toc}{section}{Заключение}

В ходе выполнения дипломного проекта была успешно решена комплексная задача анализа, проектирования и разработки интегрированной системы управления потоками пользовательских обращений. Основной целью являлось создание интеллектуального решения, способного повысить эффективность обработки запросов, снизить операционные издержки и улучшить качество взаимодействия с пользователями за счет автоматизации и предоставления прозрачных, объяснимых ответов.

Проведенный анализ существующих подходов и коммерческих решений выявил актуальную потребность в системах, которые не просто автоматизируют рутинные операции, но и обеспечивают глубокую интеграцию с верифицированными базами знаний, минимизируя риски, связанные с неконтролируемой генерацией контента большими языковыми моделями. В ответ на эти вызовы была предложена и реализована концепция системы, где LLM выступает в роли интеллектуального навигатора по иерархической базе знаний, а не прямого генератора ответов.
Ключевым результатом проделанной работы является создание программного прототипа:


\begin{enumerate}[label=\arabic*.]
    \item Универсальное ядро системы, разработанное на TypeScript с использованием среды Deno. Это ядро реализует алгоритм последовательного интеллектуального поиска по иерархической базе знаний, хранящейся в графовой СУБД Neo4j, и управляет взаимодействием с LLM через OpenAI-совместимый API.
    \item Механизмы итеративного поиска, формирования структурированных и объяснимых ответов с указанием пути их получения, а также управления контекстом пользовательской сессии.
    \item Два демонстрационных пользовательских интерфейса (адаптера): Telegram-бот для оперативного диалогового доступа и многофункциональное web-приложение на React, предоставляющее расширенные возможности визуализации процесса поиска, истории запросов и администрирования базы знаний.
\end{enumerate}

Продемонстрированные сценарии работы подтвердили способность системы эффективно обрабатывать запросы, осуществлять навигацию по базе знаний и предоставлять пользователю не только релевантный ответ, но и информацию о процессе его получения. Это повышает доверие к системе и обеспечивает прозрачность ее функционирования. Предварительные оценки указывают на значительный потенциал разработанного подхода в сокращении времени обработки типовых запросов (потенциально с десятков минут до 10-30 секунд).

Созданный прототип обладает высоким потенциалом для дальнейшего развития. Модульная структура ядра, четко определенные интерфейсы взаимодействия и выбранный технологический стек (TypeScript, Deno, React, Neo4j) закладывают прочную основу для расширения функциональности, масштабирования и возможного перехода к микросервисной архитектуре. Перспективными направлениями являются дальнейшее наполнение базы знаний, тонкая настройка промптов для LLM, интеграция с другими каналами взаимодействия, а также внедрение более сложных интеллектуальных механизмов, таких как адаптивное обучение и проактивное предложение решений.

Таким образом, дипломный проект в полной мере достиг поставленных целей, продемонстрировав жизнеспособность и эффективность предложенной концепции. Разработанные решения и программный прототип представляют собой значимый шаг к созданию интеллектуальных систем поддержки нового поколения, способных трансформировать процессы взаимодействия с пользователями в различных предметных областях, обеспечивая высокую скорость, точность и прозрачность.