\section{Экономическое обоснование разработки системы мониторинга и управления производственными линиями}
\label{sec:economics}

\subsection{Характеристика программного средства}

Целью дипломного проекта является разработка интегрированной
системы управления потоками пользовательских обращений.

Целью разработки системы интегрированной автоматизированной
обработки запросов является оптимизация процессов обработки
пользовательских обращений, сокращение времени ответа на запросы,
повышение точности предоставляемой информации, а также снижение
нагрузки на персонал за счет автоматизации рутинных операций и
интеллектуального анализа поступающих вопросов.

Преимуществами данного программного средства являются
универсальность архитектуры, возможность адаптации к различным
предметным областям, структурированное хранение знаний, наличие
механизма уточняющих вопросов, способность взаимодействия с внешними
системами, а также масштабируемость и гибкость в настройке под
специфические требования организации.

Данное программное средство будет использоваться в компании СП
«Модсен» ООО для повышения качества обслуживания клиентов, ускорения
обработки внутренних и внешних запросов, снижения операционных затрат и
оптимизации использования человеческих ресурсов в службе поддержки.

Основные функции, которые выполняет программное средство:
\begin{itemize}
    \item анализ пользовательских запросов и навигация по иерархической базе
знаний;
    \item формирование уточняющих вопросов для определения точной
потребности пользователя;
    \item предоставление релевантной информации из структурированного
хранилища знаний;
    \item взаимодействие с внешними системами для получения актуальных
данных;
    \item инициирование необходимых действий в зависимости от контекста
запроса.
\end{itemize}

Экономическая оценка целесообразности инвестиций в разработку и использование программного средства осуществляется на основе расчета и оценки следующих показателей: чистый дисконтированный доход, рентабельность инвестиций и срок окупаемости инвестиций.

\subsection{Расчет затрат на разработку и определения цены программного средства}

Для разработки системы мониторинга и управления производственными линиями, которая создается для собственных нужд, компания-разработчик составила документ на разработку программного средства. В документе определены требования к разрабатываемому программному средству и его стоимость. Стоимость программного средства рассчитана на основе полных затрат на разработку и включает следующие статьи: основная заработная плата разработчиков, дополнительная заработная плата разработчиков, отчисления на социальные нужды, прочие расходы, общая сумма затрат на разработку.

\subsubsection{Расчет затрат на основную заработную плату разработчикам}
Первым этапом расчета затрат на разработку системы является расчет оплаты команды разработчиков. 
Расчет осуществляется исходя из состава и численности команды, размера месячной заработной платы каждого участника команды, а также трудоемкости работ, выполняемых при разработке программного средства отдельными исполнителями по формуле:
\begin{equation}
    \mbox{З}_{\mbox{о}}=\mbox{К}_{\mbox{пр}}\cdot \sum_{i=1}^{n}{\mbox{З}_{\mbox{ч}i}\cdot t_{i}},
\end{equation}



    где	$\mbox{К}_{\mbox{пр}}$ -- коэффициент премий (равный 1,5); 
    
    $n$ -- категории исполнителей, занятых разработкой программного средства;
    
    $\mbox{З}_{\mbox{ч}i}$ -- часовая заработная плата исполнителя i-й категории, р.; 
    
    $t_{i}$ --трудоемкость работ, выполняемых исполнителем i-й категории, определяется исходя из сложности разработки программного обеспечения и объема выполняемых им функций, ч.
    

На 2025 год расчетная норма рабочего времени для пятидневной рабочей
недели составляет 168 часов, 8 часов работы в день, среднемесячная
расчетная норма рабочего времени – 21 день.

Разработкой программного средства занимался техник-программист.
Часовая заработная плата исполнителя определялась путем деления его
месячной заработной платы на количество рабочих часов в месяце. 
Расчет основной заработной платы представлен в таблице \ref{tab1}.

\begin{table}[!h!t]
\caption{Расчет основной заработной платы }
\label{tab1}
\centering

	\begin{tabular}{
			| >{\centering\arraybackslash}m{0.17\textwidth}
			| >{\centering\arraybackslash}m{0.13\textwidth}
			| >{\centering\arraybackslash}m{0.17\textwidth}
			| >{\centering\arraybackslash}m{0.20\textwidth}
			| >{\centering\arraybackslash}m{0.12\textwidth}|}

\hline
Категория исполнителя & Месячный оклад, р. & Часовой оклад, р. & Трудоемкость работ, ч. & Итого, р. \\ 

\hline
Инженер-программист & 4120  & 24,52   & 520  & 12750,4 \\    

\hline
Техник-программист  & 2360 & 14,04 & 410 & 5756,4\\ 

\hline
\multicolumn{4}{|l|}{Итого}    & 18506,8   \\ 

\hline
\multicolumn{4}{|l|}{Премия и иные стимулирующие выплаты (20\%)}    & 4287,66   \\ 
\hline

\multicolumn{4}{|l|}{\begin{tabular}[c]{@{}l@{}}Основная заработная плата разработчиков (Зо)\end{tabular}}  & 22794,46   \\ 
\hline
\end{tabular}
\end{table}

\subsubsection{Расчет затрат на дополнительную заработную плату разработчикам}
Дополнительная заработная плата ‒ это оплата за сверхурочный труд, различные трудовые успехи и надбавки за особые условия труда команды и включает выплаты, предусмотренные законодательством о труде, и определяется по нормативу в процентах (составляет 20\%) к основной заработной плате по следующей формуле:
\begin{equation}
\label{Zd}
    \text{З}_{\text{д}} = \frac{\text{З}_{o}\cdot H_\text{д}}{100\%},
\end{equation}


где $H_\text{д}$ -- норматив дополнительной              заработной платы(20 \%);

   $\text{З}_{\text{o}}$ -- затраты на основную заработную плату, (р.);




Подставим значение в формулу (\ref{Zd}) и вычислим $\text{З}_{\text{д}}$:

$$
\text{З}_{\text{д}} = \frac{22794,46 \cdot 20\%}{100\%} = 4558,89 \text{ р.}
$$

Согласно расчетам, затраты на дополнительную заработную плату разработчикам составит 4558,89 рублей.

\subsubsection{Расчет отчислений на социальные нужды. }
В расчете отчислений на социальные нужды отражаются обязательные отчисления по установленным законодательством тарифам в фонд социальной защиты населения, а также расходы предприятия на обязательное социальное медицинское страхование некоторых категорий работников в соответствии с законодательством. Расчет размера отчислений в фонд социальной защиты населения и на обязательное страхование определяется в соответствии с действующими законодательными актами Республики Беларусь и вычисляется по формуле:

\begin{equation}
\label{Rsoc}
    \text{Р}_{\text{соц}} = \frac{(\text{З}_{o} + \text{З}_{\text{д}})\cdot H_\text{соц}}{100\%},
\end{equation}
 
где $H_\text{соц}$ -- норматив отчислений на социальные нужды, \%.

Согласно законодательству Республики Беларусь, отчисления на социальные нужды составляют 34\% в фонд социальной защиты и 0,6\% на обязательное страхование. Подставим результаты вычислений в формулу (\ref{Rsoc}) и вычислим $\text{Р}_{\text{соц}}$:
 
$$
 \text{Р}_{\text{соц}} = \frac{(22794,46 + 4558,89)\cdot (34\% + 0,6\%)}{100\%} = 9464,26 \text{ р.}
$$

Согласно расчетам, размер отчислений в фонд социальной защиты и на обязательное страхование составляет 9464,26  рублей.
\subsubsection{Расчет затрат на прочие расходы}
Прочие расходы связаны с функционированием организации-разработчика в целом, например: затраты на аренду офисных помещений, отопление, освещение, амортизацию основных производственных фондов и так далее. При расчете данной статьи затрат учитывается норматив прочих затрат в целом по организации. В данном случае норматив прочих затрат равен 20\%. Размер затрат на прочие расходы рассчитывается по формуле:

\begin{equation}
\label{Rpr}
    \text{Р}_{\text{пр}} = \frac{\text{З}_{o} \cdot \text{Н}_{\text{пз}}}{100\%},
\end{equation}

где $\text{Н}_{\text{пз}}$ -- норматив прочих затрат в целом по организации, 20 \%.

Подставим значение из выражения в формулу (\ref{Rpr}) и произведем расчет $\text{Р}_{\text{пр}}$:

$$
 \text{Р}_{\text{пр}} = \frac{22794,46 \cdot 20\%}{100\%} = 4558,89 \text{ р.}
$$

Согласно расчетам, размер затрат на прочие расходы составляет
4558,89 рублей.

\subsubsection{Расчет суммы затрат на разработку}

Общая сумма затрат на разработку рассчитывается путем суммирования основной заработной платы, дополнительной заработной платы, отчислений на социальные нужды, прочих затрат. Формула расчета имеет следующий вид:
\begin{equation}
\label{Zp}
    \text{З}_{\text{р}} = \text{З}_{\text{о}} + \text{З}_{\text{д}} + \text{Р}_{\text{соц}} + \text{Р}_{\text{пр}},
\end{equation}


Подставим результаты вычислений в формулу (\ref{Zp}) и произведем расчет Зр:
$$
 \text{З}_{\text{р}} = 22794,46+4558,89 +9464,26+4558,89 = 41 376,5  \text{ р.}
$$

Согласно расчетам, сумма затрат на разработку составляет 41 376,5 рублей

\subsubsection{Результаты расчета цены на разработку программного средства}

В данном подразделе были рассчитаны необходимые статьи для расчета затрат на разработку, а именно: основная заработная плата разработчиков, дополнительная заработная плата разработчиков, отчисления на социальные нужды, прочие расходы и всего затрат на разработку. Результаты расчетов представлены в таблице \ref{tab2}.

\begin{table}[!h!t]
\centering
\caption{Результаты расчета цены на разработку программного средства}
\label{tab2}
\begin{tabular}{|l|c|}
\hline
\multicolumn{1}{|c|}{Наименование статьи затрат}                                                                       & Сумма, р. \\ \hline
Основная заработная плата разработчиков                                                           & 18506,8      \\ \hline
\begin{tabular}[c]{@{}l@{}}Дополнительная заработная плата разработчиков\end{tabular}           & 4558,89      \\ \hline
\begin{tabular}[c]{@{}l@{}}Отчисления на социальные нужды\end{tabular} & 9464,26      \\ \hline
Прочие расходы                                                                                            & 4558,89      \\ \hline
Общая сумма инвестиций (затрат) на разработку                                                                         & 41376,5    \\ \hline
\end{tabular}
\end{table}


\subsection{Расчет результата от разработки и реализации программного средства}
Экономия на заработной плате и начислениях на заработную плату сотрудников за счет снижения трудоемкости работ необходимо рассчитывать по формуле:
\begin{equation}
\label{Ezp}
    \text{Э}_{\text{з.п.}} = \text{К}_{\text{пр}}\cdot ({t_{\text{р}}}^{\text{без п.с.}}-{t_{\text{р}}}^{\text{с п.с.}})\cdot \text{Т}_{\text{ч}}\cdot N_{\text{П}}\cdot  (1+\frac{\text{Н}_{\text{д}}}{100})\cdot (1+\frac{\text{Н}_{\text{соц}}}{100}), 
\end{equation}

где $N_{\text{П}}$ -- плановый объем работ, выполняемый сотрудником; 

    ${t_{\text{р}}}^{\text{без п.с.}}, {t_{\text{р}}}^{\text{с п.с.}}$ -- трудоемкость выполнения работы до и после внедрения программного средства, нормо-час; 
    
    $\text{Т}_{\text{ч}}$ -- часовой оклад (часовая тарифная ставка) сотрудника, использующего программное средство; 
    
    $\text{К}_{\text{пр}}$ -- коэффициент премий (по фактическим данным предприятия или в диапазоне 1,5-2); 
    
    $\text{Н}_{\text{д}}$ -- норматив дополнительной заработной платы, (20\%); 
    
    $\text{Н}_{\text{соц}}$ -- ставка отчислений от заработной платы, включаемых в себестоимость, (34,6\%).

Подставим результат вычисления в формулу (\ref{Ezp}) и произведем расчет $\text{Э}_{\text{з.п.}}$:
$$
    \text{Э}_{\text{з.п.}} = 1,2 \cdot (1-0,6)\cdot 16\cdot 2016 \cdot  (1+\frac{20}{100})\cdot (1+\frac{34,6}{100}) =  25007,95 \text{ р.}
$$



Экономия на заработной плате и начислениях на заработную плату в результате сокращения численности работников определяется по формуле:
\begin{equation}
    \text{Э}_{\text{з.п.}}^{\text{п}} = \sum\limits_{i=1}^{n} \Delta \text{Ч}_\text{i} \cdot \text{З}_\text{i}  \cdot (1+\frac{\text{Н}_{\text{д}}}{100})\cdot (1+\frac{\text{Н}_{\text{соц}}}{100}),
\end{equation}

где $n$ -- категории работников, высвобождаемых в результате внедрения программного средства; 

   $\Delta \text{Ч}_\text{i}$ -- численность работников i-й
категории, высвобожденных после внедрения программного средства,
чел.;

    $\text{З}_\text{i}$ -- годовая заработная плата высвобожденных работников i-й категории после внедрения программного средства, р.;
    
    $\text{Н}_{\text{д}}$ -- норматив дополнительной заработной платы, (20\%); 
    
    $\text{Н}_{\text{соц}}$ -- ставка отчислений от заработной платы, включаемых в себестоимость, (34,6\%).

Экономия на заработной плате и начислениях на заработную плату в результате сокращения численности работников составляет 0 р., поскольку сотрудники не были уволены.


Экономия на материальных ресурсах рассчитывается по формуле:
\begin{equation}
\label{Ezp}
    \text{Э}_{\text{м}} = \text{К}_{\text{т.р.}}\cdot ({\text{Н}_{\text{р}}}^{\text{без п.с.}}-{\text{Н}_{\text{р}}}^{\text{с п.с.}})\cdot \text{Ц}_{\text{м}}\cdot N_{\text{П}}, 
\end{equation}

где $\text{К}_{\text{т.р.}}$ – коэффициент транспортных расходов (по данным предприятия или 1,05-1,2);

${\text{Н}_{\text{р}}}^{\text{без п.с.}}, {\text{Н}_{\text{р}}}^{\text{с п.с.}}$– норма расхода материальных ресурсов при
выполнении работ сотрудниками до и после внедрения программного
средства, нат. ед.;

$\text{Ц}_{\text{м}}$ – цена за единицу материального ресурса, р.;

$N_{\text{П}}$ – плановый объем работ, выполняемых сотрудником с использованием программного средства.
$$
\text{Э}_{\text{м}} = 1,05 \cdot (6,1-5,9) \cdot 12 \cdot 120 = 403,2 \text{ р.}
$$

Экономия на материальных ресурсах равна 403,2 рубля.

Экономическим эффектом при использовании программного средства является прирост чистой прибыли, полученной за счет экономии на текущих затратах предприятия, который рассчитывается по формуле:

\begin{equation}
    \Delta \text{П}_{\text{ч}} = (\text{Э}_{\text{тек}} - \Delta\text{З}_{\text{тек}}^{\text{п.с.}}) \cdot (1-\frac{\text{Н}_{\text{п}}}{100}), 
\end{equation}

где $\text{Э}_{\text{тек}} $ – экономия на текущих затратах при использовании программного средства, р.; 

$\Delta\text{З}_{\text{тек}}^{\text{п.с.}}$ – прирост текущих затрат, связанных с использованием программного средства (затраты на сопровождение программного средства, затраты на 
интернет-трафик и т.п.), р;

$\text{Н}_{\text{п}}$ – ставка налога на прибыль согласно 
действующему законодательству 20\%.

$$
\Delta \text{П}_{\text{ч}} = ((25007,95 + 403,2) -0)\cdot (1-\frac{20\%}{100})=25410,34 \text{ р}
$$

\subsection{Расчет показателей экономической эффективности разработки и использования программного средства в организации}

Оценка экономической эффективности разработки и использования программного средства для собственных нужд зависит от результата сравнения затрат на его разработку (модернизацию, совершенствование (и полученного экономического эффекта (годового прироста чистой прибыли). 

Так как сумма инвестиций (затраты) окупятся более чем за год, то для организации рассчитывается несколько показателей экономической эффективности. Для приведения доходов и затрат к настоящему моменту времени определяется коэффициент дисконтирования по формуле:

\begin{equation}
    \alpha _{t} = \frac{1}{(1 + d)^{t - t_p}}, 
\end{equation}

где $d$ -- требуемая норма дисконта, которая по своему смыслу 
соответствует устанавливаемому инвестором желаемому уровню   
рентабельности инвестиций, доли единицы;

$t$ – порядковый номер года, доходы и затраты которого приводятся к расчетному году;

$t_p$ – расчетный год, к которому приводятся доходы и инвестиционные затраты ($t_p$ = 1). 

Норму дисконта принимаем равным ставке рефинансирования Национального банка Республики Беларусь – 9,5\%. Расчетный период составит четыре года. Таким образом, коэффициенты дисконтирования за каждый год составляют:

$$
    \alpha _{1} = \frac{1}{(1 + 0,095)^{1-1}} =  1,
$$

$$
    \alpha _{2} = \frac{1}{(1 + 0,095)^{2-1}} =  0,91,
$$

$$
    \alpha _{3} = \frac{1}{(1 + 0,095)^{3-1}} =  0,83,
$$

$$
    \alpha _{4} = \frac{1}{(1 + 0,095)^{4-1}} =  0,76.
$$

В течение первого года осуществляется разработка приложения, поэтому в первый год экономический эффект будет меньше планируемого. Для того, чтобы учесть этот факт, необходимо выяснить, сколько времени будет затрачено на разработку приложения.

Так как работа команды разработчиков осуществляется поэтапно, то затраченное время будет равно сумме трудоемкости работ команды, и составит 900 часов.


Расчет показателей эффективности инвестиций осуществляется в табличной форме (таблица \ref{tab3}).

\begin{table}[!h!t]
\centering
\caption{Расчет эффективности инвестиций (затрат) в реализацию проектного решения}
\label{tab3}

\begin{tabular}{| >{\raggedright}m{0.3\textwidth}
			| >{\centering\arraybackslash}m{0.14\textwidth}
			| >{\centering\arraybackslash}m{0.12\textwidth} 
			| >{\centering\arraybackslash}m{0.12\textwidth}
            | >{\centering\arraybackslash}m{0.12\textwidth}
   |}

\hline 
\multicolumn{1}{|c|}{\multirow{2}{*}{Показатель}}                                                              & \multicolumn{4}{|c|}{Значение расчетного периода по годам} \\ 

\cline{2-5}  
\multicolumn{1}{|c|}{}  & 1-й год          & 2-й год          & 3-й год & 4-й год          \\

\hline 
1. Прирост чистой прибыли, р.  &     13688,31        &    25410,34        &     25410,34 & 25410,34      \\

\hline 
2. Дисконтированный результат, р.  &      13688,31       &      23123,40     &      21090,58 & 19311,85     \\

\hline 
3. Инвестиции (затраты) в реализацию проектного решения, р.   &    41376,5        &      0      &       0 & 0    \\

\hline 
4. Дисконтированные инвестиции, р. &    41376,5        &      0      &       0 & 0  \\

\hline 
5. Чистый дисконтированный доход по годам, р.                                                                    & -27688,19      &     23123,40      &      21090,58 & 13376,82      \\

\hline
6. Чистый дисконтированный доход нарастающим итогом, р. & -27688,19 & -4564,79 & 16525,79 & 2992,61 \\
\hline 
 
7. Коэффициент дисконтиро¬вания, доли единицы  &     1       &      0,91      &    0,83 & 0,76   \\
\hline 
\end{tabular}
\end{table}


В данном случае дисконтированный эффект нарастающим итогом превысит дисконтированные инвестиции на второй год. Дисконтированный срок окупаемости рассчитывается по формуле:

\begin{equation}
    \text{Т}_\text{ок}(\text{РР})=\frac{\sum_{t=1}^{n}\text{З}_t}{\frac{1}{n}\cdot\sum_{t=1}^{n}\Delta\text{П}_{\text{ч}t}}
\end{equation}

Таким образом, дисконтированный срок окупаемости равен
$$
\text{Т}_\text{ок}(\text{РР})=\frac{31452,9}{\frac{1}{4}\cdot(9352,74+16895,28+16895,28+16895,28)} = 2,1 \text{ года.}
$$

Индекс доходности инвестиций рассчитывается по формуле:
\begin{equation}
    \text{ИД}_{PI}=\frac{\sum_{t=1}^{n}\Delta\text{П}_{\text{ч}i}\cdot \alpha _i}{\sum_{t=1}^{n}\cdot \text{З}_i \cdot \alpha _i}
\end{equation}

Таким образом, индекс доходности инвестиций равен:
$$
\text{ИД}_{PI}=\frac{9352,74+15374,70+14023,08+12840,41}{31452,9} = 1,64
$$

\subsection{Вывод}

Разработка системы интегрированной системы управления потоками
пользовательских обращений направленна на повышение надежности и
отказоустойчивости IT-инфраструктуры компании «Модсен» ООО.
Внедрение данного программного средства позволит сократить трудозатраты
на отбраковку продукции, мониторинг состояния оборудования, а также
выявления причин аварийных ситуаций, что в совокупности повысит
эффективность производства.

Расчет инвестиций в разработку показал, что основная заработная плата
разработчиков составит 22794,46 руб., а общая сумма затрат с учетом дополнительных выплат, отчислений на социальные нужды и прочих расходов достигнет 41376,5 руб. Экономический эффект от использования системы, рассчитанный на основе экономии текущих затрат и прироста чистой прибыли, составил 25410,34 руб., что подтверждает целесообразность инвестиций.

По итогам экономического обоснования были получены следующие результаты:
\begin{enumerate}
\item[1.] Общая стоимость затрат на разработку системы мониторинга и управления производственными линиями составила 41376,5 рублей.
\item[2.] Прирост чистой прибыли от внедрения программного средства достиг 25410,34 рублей.
\item[3.] Инвестиции окупятся за 1,85 года (примерно 675,25 дней), а индекс доходности инвестиций составил 1.87, что свидетельствует об эффективности проекта.
\end{enumerate}

В результате экономического обоснования разработки системы мониторинга и управления производственными линиями, видно, что разработка данного программного средства является эффективной и вложение инвестиций в разработку целесообразно.