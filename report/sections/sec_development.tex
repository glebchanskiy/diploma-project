\section{РАЗРАБОТКА ИНТЕЛЛЕКТУАЛЬНОЙ СИСТЕМЫ УПРАВЛЕНИЯ ПОТОКАМИ ПОЛЬЗОВАТЕЛЬСКИХ ОБРАЩЕНИЙ}
\label{sec:development}

Этот раздел посвящен описанию процесса реализации интеллектуальной системы управления обращениями, основанной на разработанной архитектуре и механизмах.

\subsection{Общие сведения о реализации}

*   Краткий обзор стека технологий (языки, фреймворки, СУБД, брокер сообщений и т.д.).
*   Краткое описание структуры исходного кода проекта (основные каталоги, модули на уровне кода).

\subsection{Реализация архитектуры системы}

*   Пару слов про тип архитектуры

\subsubsection{Структура микросервисов}

*   Описание реализованных сервисов/модулей и их зоны ответственности.

\subsubsection{Организация асинхронного взаимодействия}

*   Детали реализации взаимодействия через Kafka. Структура топиков

\subsubsection{Реализация универсального API взаимодействия}

*   Спецификация API Возможно стоит прямо описать все его возможности.
*   Реализация точек входа для пользовательских адаптеров.

\subsection{Реализация подсистемы управления знаниями}

*   Пару слов про бз и работу с ней

\subsubsection{Модель хранения иерархической базы знаний}

*   Выбор и обоснование СУБД для хранения знаний.
*   Структура таблиц/документов для представления иерархии и связей.
*   Механизмы загрузки и обновления базы знаний.

\subsubsection{Реализация хранилища контекстной информации}

*   Выбор и обоснование технологии (например, Redis).
*   Структуры хранения данных

\subsection{Реализация ядра интеллектуальной обработки запросов}

*   Пару слов про поиск

\subsubsection{Реализация алгоритма последовательного интеллектуального поиска}

*   Описание имплементации основного алгоритма навигации по графу знаний.
*   Обработка пользовательских запросов для уточнения поиска.
*   Реализация логики выбора наиболее релевантной ветви/решения.

\subsubsection{Интеграция с языковыми моделями (LLM)}

*   Особенности подключения и использования LLM (например, через API).
*   Как LLM используются для интерпретации запросов и навигации (не генерации ответов).

\subsubsection{Формирование и объяснение ответов}

*   Механизм генерации структурированного ответа на основе найденного решения.
*   Реализация отображения пути в иерархии знаний как обоснования ответа.

\subsubsection{Реализация интеллектуальных механизмов автоматизации}
*   Если будет очень сильное желаени
*   Механизм экскалации и другие интеллектуальные штуки конкретно для ОБРАЩЕНИЙ

\subsection{Реализация пользовательских интерфейсов (Адаптеры)}

*   Пару слов про адаптер и необходимость в лёгкой возможности интегрирования

\subsubsection{Реализация адаптера для Telegram-бота}

*   Пару слов про интеграцию ядра в бота

\subsection{Тестирование и отладка}
*   Описание процесса тестирования
*   Примеры тестовых сценариев для ключевых моментов



\subsection{Результаты разработки и демонстрация работы}
*   Описание финальной версии разработанного программного продукта.
*   Демонстрация работы системы на примерах со скринами/логами
*   Обсуждение достигнутых показателей, например увеличени скорости обработки типовых запросов, точность прогнозирования и тд.


\subsection{Вывод по разделу}
*   Резюме ключевых аспектов реализации
*   Подтверждение соответствия реализованной системы проектным решениям и требованиям
*   Оценка готовности к дальнейшему развитию
