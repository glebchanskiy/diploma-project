%Перечень условных обозначений (при необходимости) оформляется в алфавитном порядке
\sectioncentered*{Перечень условных обозначений}
\addcontentsline{toc}{section}{Перечень условных обозначений}
\label{sec:reduction}

В пояснительной записке используются следующие условные обозначения:

ИИ — Искусственный интеллект;
AI - Artificial Intelligence (Искусственный интеллект);\\
ML - Machine Learning (Машинное обучение);\\
LLM - Large Language Model (Большая языковая модель);\\
NLP — Natural Language Processing (Обработка естественного языка);\\
API — Application Programming Interface (Программный интерфейс приложения);\\
CQRS — Command and Query Responsibility Segregation (Разделение ответственности команд и запросов);\\
CRM - Customer Relationship Management (Управление взаимоотношениями с клиентами);\\
UML — Unified Modeling Language (Унифицированный язык моделирования);\\
JSON — JavaScript Object Notation (Нотация объектов JavaScript);\\
MVP — Minimum Viable Product (Минимально жизнеспособный продукт);\\
CSS — Cascading Style Sheets (Каскадные таблицы стилей);\\
URL — Uniform Resource Locator (Унифицированный указатель ресурса);\\
WS-API — Web Socket Application Programming Interface (Программный интерфейс веб-сокетов);\\
БЗ — База знаний;\\
СУБД — Система управления базами данных;\\
IT — Information Technology (Информационные технологии);\\
RAG — Retrieval-Augmented Generation (Генерация с дополненным поиском);

Левков