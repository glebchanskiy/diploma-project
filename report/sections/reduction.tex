%Перечень условных обозначений (при необходимости) оформляется в алфавитном порядке
\sectioncentered*{Перечень условных обозначений}
\addcontentsline{toc}{section}{Перечень условных обозначений}
\label{sec:reduction}

В практической работе используются следующие условные обозначения:

ИИ — искусственный интеллект;\\
AI - Artificial Intelligence (Искусственный интеллект);\\
ML - Machine Learning (Машинное обучение);\\
LLM - Large Langugae Model (Большая языковая модель)
NLP — Natural Language Processing (Обработка естественного языка);\\
API — Application Programming Interface (Программный интерфейс приложения);\\
CQRS — Command and Query Responsibility Segregation (Разделение ответственности команд и запросов);\\
CRM - Customer Relationship Management (Управление взаимоотношениями с клиентами)

