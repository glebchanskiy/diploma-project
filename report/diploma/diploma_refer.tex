\sectioncentered*{Реферат}
\thispagestyle{empty}

\MakeUppercase{Разработка интеллектуальной системы управления потоками пользовательских обращений}: дипломный проект/ Г.~А. Левков. -- Минск : БГУИР, \the\year{}, -- п.з. -- ~\pageref*{LastPage}~с., чертежей (плакатов) -- 6 л. формата А1.

\vspace{4\parsep}

%Дипломный проект выполнен на 6 листах А1 с пояснительной запиской на на~\pageref*{LastPage} страницах, без приложений справочного или информационного характера. Пояснительная записка включает 4 раздела, 27 рисунков, 3 таблицы, 9 формул, 29 литературных источников.

Целью дипломного проекта является разработка интеллектуальной системы, предназначенной для эффективного управления потоками пользовательских обращений. Система нацелена на радикальное сокращение времени обработки типовых запросов, снижение операционных затрат и повышение качества обслуживания за счет применения подхода к обработке информации, основанного на иерархической базе знаний и управляемом LLM поиске.

Предметом исследования являются методы и алгоритмы интеллектуальной обработки информации, архитектурные подходы к построению масштабируемых и адаптивных систем поддержки, а также технологии интеграции языковых моделей (LLM) для навигации по структурированным базам знаний.

Разработанная интеллектуальная система позволяет осуществлять автоматизированную обработку пользовательских запросов, предоставляя точные и объяснимые ответы на основе верифицированной базы знаний. Ключевой особенностью является использование LLM не для генерации ответов, а в качестве интеллектуального навигатора, который интерпретирует запрос пользователя и управляет процессом последовательного углубления в иерархическую базу знаний. Система обеспечивает прозрачность процесса принятия решений, предоставляя пользователю информацию о пути, по которому был найден ответ.

В первом разделе пояснительной записки проведен анализ существующих подходов и коммерческих решений в области автоматизации обработки пользовательских обращений. Выявлены их ограничения, такие как недостаточная гибкость, проблема "черного ящика" и риски, связанные с неконтролируемой генерацией ответов LLM. Обоснована актуальность разработки собственной системы, основанной на принципах прозрачности, объяснимости и управляемости.

Во втором разделе описано проектирование интеллектуальной системы. Представлена концептуальная модель, основанная на иерархической базе знаний и роли LLM как навигатора. Детально проработаны варианты использования системы пользователями и операторами. Разработана абстрактная модель данных для ядра системы и ее конкретизация для задачи управления обращениями. Описаны интеллектуальные механизмы обработки, включая автоматическую обработку и прогнозирование эскалаций, а также предложены архитектурные решения, обеспечивающие модульность, масштабируемость и асинхронное взаимодействие компонентов.

В третьем разделе изложены этапы практической реализации системы. Описан выбранный технологический стек (TypeScript, Deno, Neo4j, OpenAI API), структура исходного кода. Детально рассмотрена реализация интеллектуального ядра: алгоритм последовательного поиска по графу знаний, управляемый LLM; интеграция с языковыми моделями для интерпретации запросов и навигации; механизмы формирования структурированных и объяснимых ответов с указанием пути поиска. Также описана реализация адаптера для Telegram-бота как примера пользовательского интерфейса.

В четвертом разделе приведено технико-экономическое обоснование целесообразности разработки и внедрения программного продукта, включая расчет затрат и оценку потенциального экономического эффекта.

Результатом дипломного проектирования является программный прототип интеллектуальной системы управления потоками пользовательских обращений, включающий универсальное ядро для навигационного поиска по иерархической базе знаний и демонстрационный Telegram-бот. Разработанное ядро обладает высоким потенциалом для адаптации к различным предметным областям, требующим точной и объяснимой обработки информации.

\clearpage